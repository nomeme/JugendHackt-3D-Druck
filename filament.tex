\subsection{Was ist Filament?}
{
\usebackgroundtemplate{%
\colorbox{BackgroundJGH}{%
\vbox to \paperheight{\vfil\hbox to \paperwidth{\hfil\includegraphics[width=\paperwidth]{images/filament/filament3.jpg}\hfil}\vfil}
}
}
\begin{frame}
  \frametitle{Was ist Filament?} \pause
  \begin{itemize}
    \item Ein faden aus Thermoplastischem Kunststoff \pause
    \item Je nach Drucker mit verschiedenen Durchmessern \pause
    \item Wird auf einer Spule aufgerollt verkauft
  \end{itemize}
\end{frame}
\begin{frame}
  \frametitle{Physikalische Eigenschaften}
  \pause
  \begin{itemize}
    \item Festigkeit \pause
    \item Flexibilität \pause
    \item Haltbarkeit \pause
    \item Schrumpf und Verzug (Warping) \pause
    \item Löslich (Stützstrukturen)
  \end{itemize}
\end{frame}

\begin{frame}
  \frametitle{Verarbeitungseigenschaften}
  \pause
  \begin{itemize}
    \item Durcktemperatur \pause
    \item Glastemperatur/Druckbetttemperatur
  \end{itemize}
\end{frame}
}
{
\usebackgroundtemplate{%
\colorbox{BackgroundJGH}{%
\vbox to \paperheight{\vfil\hbox to \paperwidth{\hfil\includegraphics[width=\paperwidth]{images/filament/filament4.jpg}\hfil}\vfil}
}
}
\begin{frame}
  \frametitle{PLA}
  \pause
  \begin{itemize}
    \item Das am häufigsten verwendete Filament \pause
    \begin{itemize}
      \item Biokompatibel
      \item Anwendung in bio Plastiktüten, etc. \pause
    \end{itemize}
    \item Festes, sprödes Material, bricht leicht
  \end{itemize}
\end{frame}

\begin{frame}
  \frametitle{Eigenschaften}
  \pause
  \begin{itemize}
    \item \textbf{Schwierigkeit:} Gering
    \item \textbf{Drucktemperatur:} \unit[180 - 230]{°C}
    \item \textbf{Druckbett-Temperatur:} \unit[20 - 60]{°C}
    \item \textbf{Schrumpf und Verzug:} Gering
    \item \textbf{Haltbarkeit:} Durchschnittlich
    \item \textbf{Glastemperatur:} \unit[45-65]{°C}
    \item \textbf{Löslich:} Nein
  \end{itemize}
\end{frame}

\begin{frame}
  \frametitle{ABS}
  \pause
  \begin{itemize}
    \item Robuster Kunststofft \pause
    \begin{itemize}
      \item Eignet sich zum Beschichten mit Metallen und anderen Kunststoffen
      \item LEGO-Steine, Playmobil, Motorradhelme \pause
    \end{itemize}
    \item Festes, haltbares und temperaturbeständiges Material
  \end{itemize}
\end{frame}

\begin{frame}
  \frametitle{Eigenschaften}
  \pause
  \begin{itemize}
    \item \textbf{Schwierigkeit:} Hoch
    \item \textbf{Drucktemperatur:} \unit[210-250]{°C}
    \item \textbf{Druckbett-Temperatur:} \unit[80 - 110]{°C}
    \item \textbf{Schrumpf und Verzug:} Stark
    \item \textbf{Haltbarkeit:} Hoch
    \item \textbf{Glastemperatur:} \unit[95 - 110]{°C}
    \item \textbf{Löslich:} Ester, Ketonen und Aceton
  \end{itemize}
\end{frame}

\begin{frame}
  \frametitle{PETG}
  \pause
  \begin{itemize}
    \item Lebensmittelsicherheit fraglich (hormonaktive Eigenschaften?) \pause
    \begin{itemize}
      \item Anwendung in PET Flaschen
      \item Teil-biobasiert erhältlich \pause
    \end{itemize}
    \item Festes, flexibles, haltbares Material \pause
    \item Extrem hygroskopisch und klebrig (Stützstrukturen)
  \end{itemize}
\end{frame}

\begin{frame}
  \frametitle{Eigenschaften}
  \pause
  \begin{itemize}
    \item \textbf{Schwierigkeit:} Gering
    \item \textbf{Drucktemperatur:} \unit[220 - 250]{°C}
    \item \textbf{Druckbett-Temperatur:} \unit[50 - 75]{°C}
    \item \textbf{Schrumpf und Verzug:} Gering
    \item \textbf{Haltbarkeit:} Hoch
    \item \textbf{Glastemperatur:} \unit[70]{°C}
    \item \textbf{Löslich:} Nein
  \end{itemize}
\end{frame}

\begin{frame}
  \frametitle{TPU}
  \pause
  \begin{itemize}
    \item Eine Form der Polyurethane \pause
    \begin{itemize}
      \item Anwendung(PU): Schaumstoffe, Lacke, Beschichtungen, Klebstoffe, Vergussmassen
      \item TPU hat fast gummiartige Eigenschaften \pause
    \end{itemize}
    \item Wegen der Materialeigenschaften schwer zu drucken \pause
    \item Extrem hygroskopisch
  \end{itemize}
\end{frame}

\begin{frame}
  \frametitle{Eigenschaften}
  \pause
  \begin{itemize}
    \item \textbf{Schwierigkeit:} Mittel
    \item \textbf{Drucktemperatur:} \unit[210 - 230]{°C}
    \item \textbf{Druckbett-Temperatur:} \unit[30 - 60]{°C}
    \item \textbf{Schrumpf und Verzug:} Gering
    \item \textbf{Haltbarkeit:} Sehr hoch
    \item \textbf{Glastemperatur:} \unit[-223]{°C}
    \item \textbf{Löslich:} Nein
  \end{itemize}
\end{frame}
}
